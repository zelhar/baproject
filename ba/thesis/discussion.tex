\section{Discussion}

RWR propagation methods have been successfully used in
bioinformatics mainly as a component in algorithms for 'disease
module prediction', for example by ~\cite{leiserson2015pan,
barel2020netcore}. More generally diffusion based algorithms have
been used in multiple applications~\cite{cowen2017network}.

The prediction method which was used by
\textcite{schwikowski2000network} is essentially method 3 but their
test setup allowed multiple labels per proteins. Their experimental accuracy
was 72\%.

We adopted the PLC name and definition from
\textcite{szummer2002partially}. The proposed solution for the PLC
on that paper was to maximize the posterior probability of the label
for each unlabeled vertex. Essentially find label $y$ which is the likeliest to
be the group where the walk had began, considering it ended at vertex $k$ after
$t$ steps. Their solution uses a fixed time scale but it doesn't have to. We
think that method 5 is possibly the solution if we use RWR instead of time scale.
That is because method 5 chooses for vertex the label from the group that if we
start from it, it would be visited more frequently than if we started from any
other labeled group. The fact that we deal with stationary distribution saves
allot of headache inducing calculations but of course we need to justify why do
we allow restart.

Another paper which deals with protein function prediction is
\textcite{deng2002prediction}. Like in \cite{schwikowski2000network}
they allowed multiple labels per protein. They measured
sensitivities and specificities which were 'rougly the same' and depending
on the setting were 45\% to 65\%. But we don't think those results
are comparably with the experiment that we presented here.

An idea of improving method2 would be to assess the likelihood of
the correctness of the prediction for a vertex on the fly, and only
include the preicted labeling in the prediction of the follow up
prediction if the likelihood is high enough. In this case we might
alsop try to run it many times in different random orders rather
than in a particular order.

Assessment of the likelihood of the prediction is very important in
the cases where we don't want or need to predict the labels of all
the unknown proteins, but rather need to select specific unlabeled
proteins which are very likely to have a specific label we are
interested in. Essentially this is what disease module prediction is about.

%Propagation methods or as they are sometimes called, diffusion methods, have
%been used in Bioinformatic software tools mostly function prediction and
%'disease characterization' \cite{cowen2017network}.
%
%Typically the diffusion method is used to screen for possible 'candidates' (i.e
%a disease causing genes) which is basically pageRanking, or for module detection
%which is essentially clustering/community structure. In a second step these
%candidates are scored by a probabilistic model which may use additional
%information beyond the network structure itself.
%
%The most simple application, which is briefly described in the introduction, is 
%function prediction in a single PPI network. As an example application we bring
%the RIDDLE \cite{wang2012riddle} functional prediction web application
%(\url{www.functionalnet.org/RIDDLE}). 
%
%(usw und sofort \dots they use pagrRank, and then do a statistical test based on
%hypergeometric distribution to score the significance \dots) 
%
%Hotnet \cite{vandin2012discovery} and its successor Hotnet2
%\cite{leiserson2015pan} are an example of a more complex application
%used for module detection \dots
%
%(they have a system to set threshold influence, break the graph to small 
%connected components as module candidates, then use additional information 
%to score the significance)
