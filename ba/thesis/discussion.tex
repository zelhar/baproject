\section{Further Discussion and Remarks}
%The normalize adjacency matrix of 
%a connected graph (for simplicity we use here undirected) is a transition matrix
%and represents a
%Markov process. On a given time-step, a visitor on a graph node chooses
%his next station at random among the neighbours of the current station (node).
%
%We want to know first, if we repeat this process to infinity will the frequency
%of the visits at each node stabilizes.

\lipsum[3-4]

%Propagation methods or as they are sometimes called, diffusion methods, have
%been used in Bioinformatic software tools mostly function prediction and
%'disease characterization' \cite{cowen2017network}.
%
%Typically the diffusion method is used to screen for possible 'candidates' (i.e
%a disease causing genes) which is basically pageRanking, or for module detection
%which is essentially clustering/community structure. In a second step these
%candidates are scored by a probabilistic model which may use additional
%information beyond the network structure itself.
%
%The most simple application, which is briefly described in the introduction, is 
%function prediction in a single PPI network. As an example application we bring
%the RIDDLE \cite{wang2012riddle} functional prediction web application
%(\url{www.functionalnet.org/RIDDLE}). 
%
%(usw und sofort \dots they use pagrRank, and then do a statistical test based on
%hypergeometric distribution to score the significance \dots) 
%
%Hotnet \cite{vandin2012discovery} and its successor Hotnet2
%\cite{leiserson2015pan} are an example of a more complex application
%used for module detection \dots
%
%(they have a system to set threshold influence, break the graph to small 
%connected components as module candidates, then use additional information 
%to score the significance)
