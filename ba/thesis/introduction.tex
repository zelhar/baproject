%Introduction
\section{Introduction}

In this paper we propose a protein function prediction algorithm,
which in essence works similarly to the adage ''Tell me who your
friend are, and I will tell you who you are''. Imagine that we can
accurately predict the function of a protein in the cells of an
organism bases on its biochemical interactions with the other
proteins, and the interactions of the other proteins between
themselves. This has in fact demonstrated to be true by
\textcite{schwikowski2000network}. But we are trying to improve on
it. In the formulation of the adage above, We want to tell you who
you are based on information on some friends of your friends. 
We want that because in the world of protein research, there are
still many that are unresearched or mostly unresearched.

The specific problem we are dealing here, from the perspective of
bioinformaticians, is of protein function prediction in a
protein-protein interaction network (PPIN), yet the actual algorithm
is more general and applies to any network really. For this reason
why we call the problem 'Partially labeled classification' (PLC) as it was
named by \textcite{szummer2002partially}.

Our proposed algorithm uses the network propagation method of random
walk with restart (RWR). These methods has been used most famously
by Google for their search algorithm but also in bioinformatics  for
example for 'disease module identification' by
\cite{vandin2012discovery, barel2020netcore}.

We dedicate sections 2 and 3 of this Thesis to explain the
underlying mathematical theories of stochastic matrices and random
walk with restart in graphs, largely going according to textbooks
from \textcite{meyer2000matrix} and \textcite{herstein_winter_1989}.
The purpose is to make this work somewhat self standing with the
mathematical foundations it is based on, to give a 'feel' for how
RWR actually work, as well as presenting some
lemmas that prove why we can use it in the networks we are dealing
with.

In the following 2 sections (4 and 5) We deal with the subjects
of community structure and spectral clustering in graphs. There is
a great deal of overlap between the PLC problem and community structure. 
Essentially the function prediction algorithm is based on the
paradigm that proteins of the same functional classification do tend
to form 'communities' within the PPI.
In section 6 we are propose an algorithm for protein function
prediction (or more generally for the partially labeled
classification problem), and we test it and compare it with some
other methods. 

Finally in section 7 we draw some conclusions, , discuss other
methods and approaches, and consider possible ways to carry with
this research forward.
