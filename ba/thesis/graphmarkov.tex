\section{Appendix: More on matrices, graphs and stochastics}
\subsection*{}

\subsection*{}

A directed non-weighted graph $G$ can be uniquely represented by its
\textbf{adjacency matrix},
$A_{i,j} := 1$ if and only if there is a directed edge from $j$ to $i$ (if we
want to use it for transitioning on columns as done above). It's possible to
assign different edge weights rather than only $1$ and $0$. If the graph is
undirected each edge would be counted in both directions and the matrix is
symmetric.
Relabeling the vertices of a graph yields an adjacency matrix that is similar by
permutations ($PAP^{-1}$, where $P$ is a permutation matrix) and vice versa.

To turn $A$ into a transition, normalize each column by dividing it with
its out going rank, so let $D_{i,i} = \text{out~rank}(i)$, $T:=AD^{-1}$ is the
transition matrix of this graph (because right-multiplying by $D$ normalizes each
column by its rank).
If the graph was stochastic to begin with then the adjacency matrix as we
defined it is already column-normalized.

\begin{mydef}
\label{Ax:def:stronglyconnected}
A graph is $G$ \textbf{strongly connected} if there is a directed path form any edge
to any edge. Equivalently $G$ is strongly connected if and only if its adjacency matrix is irreducible.

We say that the \textbf{Period of a vertex} $v \in V(G)$ is the greatest common
divisor of all closed paths on $v$. If the periods of all vertices are equal
(spoiler: in the case that $G$ is strongly connected they are), we call it the
\textbf{Period of the graph} $G$.
\end{mydef}

\begin{remark}
\label{Ax:remark:periods}
If $G$ is strongly connected then the periods of all vertices are indeed equal
and its easy to prove. The corresponding adjacency matrix $A$ is irreducible so
it too has a period $h$ as defined in \ref{Ax:remark:wielandt_cyclicity} and it is
equal to the graph period (can be shown using \ref{Ax:thm:wielandt2} and
\ref{Ax:remark:wielandt_cyclicity}).

So if the graph $G$ is strongly connected and has period $1$
then the adjacency matrix is \textbf{aperiodic} and hence primitive, and vice versa. 
\end{remark}

From all the above we have seen that a Markov process can be represented in two
equivalent ways \textemdash as a transition matrix  and as the 
corresponding weighted directed graph.

If the graph $G$ is strongly connected and aperiodic, its corresponding
adjacency matrix is primitive. We know from \ref{Ax:thm:perron1} that there is a
unique stationary distribution $p$ and that the Markov process converges to $p$ no
matter from which distribution it starts. We may calculate $p$ using the
\textbf{power
method} which is efficient because it can be done in a matrix-free method. 
%We don't need to know the matrix itself just the dot product of it with a state
%vector.

If the graph $G$ is strongly connected, then \ref{Ax:thm:perron2} assures us the
existence and uniqueness of a stationary distribution $p$. But if $G$ is not
aperiodic, the corresponding adjacency matrix is not primitive. We cannot use
the efficient power method to calculate $p$. Also the process itself doesn't
stabilize on $p$. It is periodic and cycles between the $h$ eigenvectors on
the unit circle (see theorem \ref{Ax:thm:wielandt2}). 

We are therefore interested to find how to convert an \text{imprimitive}
matrix (= irreducible but not primitive)
to a primitive matrix or equivalently to turn a strongly connected but periodic graph
into an aperiodic graph.

\begin{lemma}\cite{meyer2000matrix}
\label{Ax:lem:1plusA}
Let $A \geq 0$ be irreducible. Then $(A + I)^{n-1} \gt 0$, and therefore $A+I$
is primitive.
\begin{proof}
Notice that the $I$ represents self loops and it absorbs all the lower powers so
if $(A^k)_{i,j} \gt 0$ for some $k \lt n$ then so is $(A+I)^{n-11}$.
Let $G := G_A$ be the corresponding graph to $A$. Then $G$ is strongly
connected. For every $i \neq j$ there is a directed \textbf{shortest path}
$\sigma$ in $G$ from $i$ to $j$. Its length must be $|\sigma| \lt n$. Otherwise
$\sigma$ would have to contain a cycle and not be of minimal length.
This shows that we have for every $i \neq j$ some $k$ sucht that $A^k_{i,j} \gt
0$ and therefore $(A+I)^{n-1} \gt 0$.
\end{proof}
\end{lemma}

The properties of irreducibility, primitiveness and positivity only depend on the
sign ($-,0,+$) of the entries and not on their size. So we use the following definition to
test matrices for these properties:

\begin{mydef}
Let $A \in \R^{n \times n}$, then its \text{binary form} is the matrix
$\beta(A) := \text{sgn}(A)$ where $\text{sgn}$ is applied element-wise.
\end{mydef}

The following trivial lemmas would help as to construct primitive transitions:

\begin{lemma}
\label{Ax:lemma:SplusT}
$A$ is positive/primitive/irreducibly if
and only if $\beta(A)$ is.

Let $0 \leq \beta(A) \leq \beta(B)$.
Then If $A$ is positive/primitive/irreducibly so is $B$.

Let $0 \lt \alpha \lt 1$. If $T,S$ are transitions the so is $W=(1-\alpha)T +
\alpha S$.
If one of $S,T$ is positive/primitive/irreducible so is $W$
\end{lemma}

Let $G$ be any weighted graph and $A$ its adjacency transition matrix. Some vertices may
be unreachable from other vertices and there might not exist a single and
unique stationary distribution.
A random walk on this graph is generally
dependent on the initial starting distribution $p_0$ and has the
form $p_{k+1} = Ap_k = \dots = A^k p_0$, where $p_k$ is the
distribution after $k$ steps.

However if we allow the possibility of 'random restart' from any state, this
graph becomes totally connected it is guarantied to have a unique stationary
distribution to which any random walk converges regardless of the initial state.

When we talk about \textbf{random walk with restart (RWR)} we set a restart
state $q$ and a restart parameter $\alpha \in [0,1]$. At each step, we
either restart over to the state $q$, with probability of
$alpha$, or continue to walk using the normalized adjacency matrix
$A$. The state sequence is therefore
\begin{equation}
\label{Ax:eq:RWR}
p_{k+1} = \alpha q + (1 -
\alpha) A \cdot p_k
\end{equation}

It turns out that this random walk with restart is actually a normal
random walk but with an modified adjacency matrix (and respectively, an
modified weighted directional graph). 

\begin{lemma}
\label{Ax:lemma:Qq}
Let $q$ be any state, let $\alpha \in [0,1]$ and let  
$ Q = (q|q|\dots|q)$ (The square matrix whose every column equals $q$).
Then $Q$ is a transition and for any state $p$ we have $Qp = Q$.

In addition if $T$ is any transition then $W = \alpha Q + (1-\alpha)
T$ is also a transition, And we can rewrite the RWR from
\ref{Ax:eq:RWR} as

$p_{k+1} = \alpha q + (1 - \alpha) Ap_k = 
[\alpha Q + (1 - \alpha) A] p_k = W p_k$

\begin{proof}
Trivial and uses \ref{Ax:lemma:SplusT}
\end{proof}
\end{lemma}

The matrix $W$ represents a graph $G'$ where each edge of the
original graph $G$ is rescaled by a factor $1 - \alpha$ (and if $G$ is
undirected we treat each undirected edge as $2$ directed edges in
$G'$. In addition from each vertex $v$ we add edges to every other
vertex and the weight of the additional edge is $\alpha q[u]$.

If we pick the restart state $q$ in a way that makes $W$ primitive,
then \ref{Ax:thm:perron1} assures us that the RWR will converge,
$\lim_{k \to \infty} p_k =\lim_{k \to \infty} W^k p_0 = p$, where
$p$ is the unique stationary distribution of $W$. This means in
particular that we can use the power method to find out the
distribution $p$ by sequentially calculating $p_1, p_2, \dots$ until
it sufficiently converges, and it will converge to $p$ from any
initial state $p_0$ which we choose. 

Also we can take the limit $p = \lim_{k \to \infty}p_k$ and rewrite
\ref{Ax:eq:RWR} as:
\begin{equation}
\label{Ax:eq:RWR2}
\begin{aligned}
p = Wp = \alpha q + (1 - \alpha) A \\
[I - (1 - \alpha)A] p = \alpha q
\end{aligned}
\end{equation}

We will see later that we can invert the matrix in the second
equation of \ref{Ax:eq:RWR2} and use a direct solution for $p$ instead of
the power method. The power method has the advantage that we can use
the matrix $A$ implictly, because we only need to know $A \cdot p_k$
to compute $p_{k+1}$. $A$ is usually a sparse matrix because each
vertex usually only has few neighbors and so we can use matrix free
methods efficiently to calculate $p$ but that is beyong the scope of
this thesis.

In the particular case of pageRank, we choose a uniform restart
state $q = \mathbf{\frac{1}{n}}$. The corresponding matrix $Q =
(q|\dots|q) \gt 0$ is strictly positive, and therefore 
$W = \alpha Q + (1 - \alpha)A > 0$ is positive and therefore
primitive.

The stationary distribution which corresponds to this uniform
restart state $q$ is called the \textbf{PageRnak} for $G$ with
restart parameter $\alpha$. It is used to order the vertices
according to their 'relevance' in the network.



The PageRank is the stationary distribution of the process when we use unbiased
restart\textemdash A restart is equally likely from any vertex.
But we want more. We want to find out what happens when we restart, for example,
always from one single vertex $u$. We think of the stationary distribution $p_u$ that 
results from such process as the heat (or flow) which propagates out of $u$.
If we take another vertex $v$ we think of $p_u[v]$ as a measure of how close $v$
is to $u$, or how much heat $u$ sends to $v$.
Note that this is not symmetric $p_v[u] \neq p_u[v]$ in general.

%(memo: add an illustration about asymmetry)

\begin{lemma}
\label{Ax:lem:AplusP}
Let $A \geq 0$ be irreducible. Let $B \geq 0$ have a positive row (or column),
then $A + B$ is primitive.
\begin{proof}
Suppose WLG that $B_1 > 0$ (first row). Then $\forall k \gt (B^k)_1 \gt 0$.
Fix $i,j$.
Since $A$ represents a strongly connected graph, there is a path from $i
\curvearrowright 1 \curvearrowright j$ of some length $k$. So $A^k_{i,j}>0$.
Then $\forall l \geq 0 (A + B)^{k+l} \gt 0$ because we can go along this path,
then self loop $l$ times in $1$ and keep going to $j$ on that path.

So we have shown that $(A+B)^k_{i,j}$ stays positive once it becomes positive
and it always turns positive at some point by irreducibility of $A$, so that
means $A+B$ is primitive.
\end{proof}
\end{lemma}

\begin{remark}
\label{Ax:rem:AplusP}
Lemma \ref{Ax:lem:AplusP} proves that we can propagate (namely do RWR) from any arbitrary restart
state $q$, including a single vertex and the combined transition matrix will be
primitive if the adjacency matrix $A$ is irreducible, or
equivalently, the corresponding graph $G$ is strongly connected.


Assume that $A$ is a normalized adjacency transition of a strongly
connected graph $G$.
Let $q$ be any state column vector, for example $e_1$ if we restart
from a single vertex, and let $Q = (q | q | \dots | q)$.
So $Q$ has a positive row and therefore $(1-\alpha)A + \alpha Q$ is a primitive
transition according to \ref{Ax:lem:AplusP}.

%Also worth noting that if $x \geq 0$ is any transition, then $Qx = q$.
\end{remark}

\begin{mydef}
\label{Ax:def:Transitionbiased}
Let $G$ be a graph with adjacency matrix $A$, and let $D$ be the
diagonal matrix of the out ranks of the vertices of $G$. Then we can
column normalize $A$ and create the transition $T = AD^{-1}$.
We define \textbf{the transition matrix with restart parameter}
$\alpha$ \textbf{and bias} $q$ as
\[
T_{\alpha, q} :=
(1 - \alpha)T + \alpha Q
\]
\end{mydef}

In general, 
the matrix $T_{\alpha,q}$ may have more than one unique stationary distribution $p$ 
(there is one for each strongly connected
component of its corresponding graph). But if we require that $G$ be
strongly connected (which means $A$ is irreducible),
then $T_{\alpha,q}$ is primitive by \ref{Ax:lem:AplusP}, and the
stationary distribution $p$ is unique.

\[
p = Ip
= [(1 - \alpha)T + \alpha Q]p =  (1 - \alpha)Tp + q 
\]

We can rearrange it now

\begin{equation}
\label{Ax:eq:uninvertedstationary}
(I - (1 - \alpha)T)p = \alpha q
\end{equation}

We want to invert the matrix in \ref{Ax:eq:uninvertedstationary} and use the
following lemma to justify it (The proof is easy. See for example
\textcite{serre2010matrices}):

\begin{lemma}
\label{Ax:lem:invertible}
Let $X$ be a contracting matrix, Then $(I-X)$ is invertible and the power sum of
$X$ converges to it:
\[
(I - X)^{-1} = \sum_{k=0}^{\infty} X^k
\]
\end{lemma}

So now we may apply lemma \ref{Ax:lem:invertible} on equation
\ref{Ax:eq:uninvertedstationary} because $(1-\alpha)A$ is contracting, and we have:

\begin{equation}
\label{Ax:eq:diffkernel}
p = \alpha [I - (1 - \alpha)T]^{-1} q := K q = 
\alpha \sum_{k=0}^{\infty} (1 - \alpha)^k T^k
\end{equation}

$K$ is called \textbf{diffusion matrix}~\cite{leiserson2015pan} of $T$ (or of the graph $G$) with parameter $\alpha$
It turns out that $K$ itself is a transition because it maps the
arbitrary transition $q$ to the transition $p$. In addition, $K \gt 0$ because
of \ref{Ax:eq:diffkernel} and since $T$ is irreducible.

What about the eigenvectors and eigenvalues of $K$?
If a matrix $A$ is invertible then $Av = \lambda v \iff A^{-1}v = \lambda^{-1}$.
For any matrix $Av = \lambda v \iff (I + A)v = (1+\lambda)v$.

It turns out then that $T$ and $K$ have the same eigenvectors and if $Tv=\lambda
v$ then $K v = \alpha [1 - (1 - \alpha) \lambda]^{-1} v$. And in particular if it
turns out that if all the eigenvalues are real (spoiler- they are), then they have
the same linear order as eigenvalues of $K$ or $T$ for the same eigenvector.
In particular we see that the choice of $\alpha$, the restart parameter, NEITHER 
changes the eigenvectors NOR does it change the order of the eigenvalues.

To sum up the important facts that we would need later:

\begin{thm}
\label{Ax:thm:AKTcharacteristics}
Let $G$ be strongly connected undirected graph. Let $A$ be its adjacency matrix
and $D$ the diagonal matrix which has the ranks of the vertices on its diagonal.
Let $T = A D^{-1}$, let $0 \lt \alpha \lt 1$ and 
$K = \alpha [I - (1 - \alpha)T]^{-1}$. Then the following are all true:

\begin{itemize}

\item{}
$A \geq 0$ is symmetric therefore its eigenvalues are all real.

\item{}
$T = AD^{-1} = D^{1/2}[D^{-1/2}AD^{-1/2}]D^{-1/2}$ is similar to $A$. Therefore
it has the same eigenvalues as $A$, which are all real:
$\lambda_1 \geq \dots \lambda_n$.

\item{}
There exists for all $0 \lt \alpha \lt 1$ the invertible matrix: 
$K := \alpha \sum_{k=0}^{\infty} (1 - \alpha)^k T^k = \alpha [I - (1 -
\alpha)T]^{-1}$.
$K \gt 0$ is a transition. It has the same eigenvectors as $T$ and their
corresponding eigenvalues are all real and they have the same order as the
corresponding eigenvalues for $T$ have.
\end{itemize}
\end{thm}

%The Equation looks something like this: Let $L = D - A$, the \textbf{Laplacian
%Matrix}. Let $\gamma \gt 0,$ and $0 \lt p_0 \in \R^n$ 

