\section{Linear algebra primer}

\subsection*{Prologue}
This section is mostly based on \cite{meyer2000matrix} and
\cite{herstein_winter_1989}. We are going to have to use allot of
definitions anyway and cite big theorems like the Perron-Froebenius
theorems, so we might as well provide an
understanding of the important properties of stochastic matrices
which we need for the RWR methods.

This chapter is accompanied by an appendix chapter, which containes
more material and presents proofs or sketch of a proof to most
theorems and lemmas.
We try to keep the chapter intself short and just mention theorems
and properties that are strictly necessary for the following
chapters.

%This chapter together with the appendix
%is also meant to intercept likely questions, quite possibly even by
%by the author himself within a few decades from now, such as, 'what if we
%consider complex vectors, maybe there is a complex eigenvalue who is
%greater than 1 ?', no there isn't, read here why.

\subsection*{Theory}

\begin{mydef}
\label{def:transition} A \textbf{Transition Matrix } (we call
it a \textbf{Transition} in short), which is sometimes also called a
\textbf{stochastic matrix}, is a real valued non negative square
($n \times n$) matrix $A$ such that each of its columns sums up to one: $$A
\geq 0,\ \forall j \sum_i A_{i,j} = 1$$ $A$ acts from the left as a
linear mapping: $A:v \to A \cdot v$. In this paper we use left
multiplication convention ($Av$). There are many other publications
that deal with random walk and Markov processes, where right
multiplication is used ($v \cdot A$), and accordingly the rows are
normalized rather than the columns.

A transition is \textbf{positive}, designated by $A > 0$ if all its entries are positive.

A transition is \textbf{primitive} if for some $k>$ $A^k$ is positive. The same
property is called \textbf{regular} in some other sources.

A transition is \textbf{irreducible} if for every entry $A_{i,j}$ there is some $k$ such
that $A^k_{i,j} > 0$.

It can be shown that 
A matrix $A$ is 
irreducible by and only if it is NOT
similar by permutations matrices to a block upper triangular matrix
which means 
$
\nexists P : 
A =
P
\begin{pmatrix}
B & C \\
0 & D
\end{pmatrix}
P^{-1}
$

where $P$ is a permutation matrix and $B, C$ are square matrices of size greater than $0$.
\end{mydef}

\begin{mydef}
\label{def:state}
A \textbf{State} is a non-negative vector $v \in R^n$ s.t $\sum_i v_i = 1$.
\end{mydef}

\begin{remark}
\label{remark:state}
If $v$ is a
state and $A$ is a transition as defined in
\ref{def:transition}, then
  $Av$ is also a state because: 
$$\sum_i(Av)_i = \sum_j v_j(\sum_k A_{j,k}) = \sum_j v_j \cdot 1 = 1$$
Also its easy to confirm by multiplying with $e_i$, that if $Av$ is a state
for every state $v$ 
each column $A e_i$ must sum to $1$.
Therefore this is an equivalent definition for a transition.

If $A$ is a transition and $x,y$ are two states such that
$x \leq Ay$ then $x=y$.
\end{remark}

\begin{mydef}
\label{def:abs}
%changed C to R per Martin's request but it holds for C
Given $A \in \R^{n \times n}$ We let $|A| \in \R^{n \times n}$ be the resulting
matrix from applying $|\cdot|$ element wise. Given a vector $v \in \C^n$ we let
$|v| \in \R^n$ the corresponding non-negative vector.

We also let $A \gt 0, v \gt 0$ mean that it holds coordinate wise. 
\end{mydef}

Here is a very useful lemma for non-negative matrices which we will need later:
\begin{lemma}
\label{lem:eqal_by_vector}
Let $0 \leq A \leq B \in \R^{n \times n}$ 
and let $0 \lt v \in \R^n$.
If $Av = Bv$ then $A = B$.
\begin{proof}
Trivial.
\end{proof}
\end{lemma}

\begin{remark}
\label{remark:abs}
If $u \in \C^n$ is on the unit circle and $T$ a transition then
$|u|$ is a state, meaning $\||u|\|_1=1$, so $T|u|$ is also a state 
so $\|T|u|\|_1=1$.

We have (component-wise) $|Tu| \leq T|u|$. If $T>0$ and $u$ has negative or
non-real entries, then this inequality must be strict and
then $\|Tu\|_1 \lt \|T|u|\|_1 = 1$.
\end{remark}

\begin{lemma}
\label{lem:exist1}
If $T$ is a transition, then
there is a state $u$ such that $Tu = u$.
\begin{proof}
In the appendix
\end{proof}
\end{lemma}


\begin{lemma}
\label{lem:uniq1}
If a transition $A \gt 0$ (or primitive)
, then it has exactly
one eigenvector $v$ with eigenvalue $1$ and in addition $v$ can be
chosen to be strictly positive.
Furthermore, for any other eigenvalue
$\lambda$ it holds that $|\lambda| \lt 1$.

If $A$ is just irreducible then then again $v>0$ and is unique but there may be
additional eigenvalues on the unit circle.

\begin{proof}
In the appendix
\end{proof}
\end{lemma}

\begin{remark}
\label{remark:rhoisone}
The lemmas and theorems in this section are phrased in term of transitions. They
hold true in the more general case of positive/non-negative
linear transformations and one just replaces $1$ with the \textbf{spectral
radius} $ \rho = \rho(A)$.

In general a non-negative linear transformation has a \textbf{spectral radius}
$\rho = \rho(A)$ which is the absolute value of its greatest eigenvalue. In the case of
positive maps there is a unique single eigenvector with $\rho$ as the unique
greatest eigenvalue and so forth \dots. When we deal with a transition map,
lemma \ref{lem:uniq1} guaranties 
it has a spectral value of $\rho(A) = 1$.
\end{remark}

\begin{thm}
\label{thm:transition_ev}
Let $T$ be a positive (or primitive) transition. Then 

1. $1$ is the greatest eigenvalue
of $T$ and it has one unique eigenvector which is also positive,
so there exists a unique stationary state.

2. All the other eigenvalues have absolute value strictly less than $1$.

3. For every state $u$, $T^ku$ converges to the stationary state $\pi$.
In particular the columns of $T^k$ converge to $\pi$.

\begin{proof}
In the appendix
\end{proof}
\end{thm}

\begin{thm}
\label{thm:transition_irr_ev}
Let $T$ be an irreducible positive (or primitive) transition. 
Then:

1. Then $1$ is the greatest eigenvalue
of $T$ and it has one unique eigenvector which is also positive,
so there exists a unique stationary state.

2. If there are other other eigenvalues on the unit circle then their algebaic
multiplicity is equal their geometric multiplicity.

3. For every state $u$, the \textbf{Cesaro sums} 
$\frac{1}{n}\sum_{k=1}^n T^ku$ converge to the stationary state $\pi$.

\begin{proof}
In the appendix
\end{proof}
\end{thm}

What differs irreducible non-primitive matrices from primitive is
that they are periodical on their eigenvectors with complex eigenvalues
on the unit cycle. There is, in fact a wonderful theorem from Wielandt which
characterizes these Matrices, which is stated with a sketch of proof
in the appendix.

Now we will just present the Perron-Frobenius theorems. The main parts that are
important to our work have appeared in the previous theorems.

\begin{thm}[Perron-Frobenius \cite{meyer2000matrix}]
\label{thm:perron1}

Let $0 \lt A \in \R^{n \times n}$ with spectral radius $\rho := \rho(A)$, then the following are all true:
\begin{itemize}
\item{} $\rho \gt 0$
\item{} $\rho$ is a simple root of the characteristic polynomial of $A$,
in other words its algebraic multiplicity is $1$.
\item{} $(\exists v > 0) Av=\rho v$
\item{} If $Au = \lambda u$ and $\|u\|= \rho$ then $\lambda = \rho$
namely, $\rho$ is the unique eigenvalue on the spectral circle.
\item{(Collatz-Wielandt Formula)} $\rho = \max_{x \in \Gamma} \min_{i : x_i \neq 0} [Ax]_i / x_i$
with $\Gamma = \{x | x \geq 0, x \neq 0\}$
\end{itemize}
\end{thm}

\begin{thm}[Perron-Frobenius for irreducible matrices \autocite{meyer2000matrix}]
\label{thm:perron2}

Let $0 \leq A \in \R^{n \times n}$ be irreducible with spectral radius $\rho := \rho(A)$,
then the following are all true:
\begin{itemize}
\item{} $\rho \gt 0$
\item{} $\rho$ is a simple root of the characteristic polynomial of $A$,
in other words its algebraic multiplicity is $1$.
\item{} $(\exists v > 0) Av=\rho v$
\item{} There are no additional non-negative unit eigenvectors of $A$ other than
$v$. 
\item{(Collatz-Wielandt Formula)} $\rho = \max_{x \in \Gamma} \min_{i : x_i \neq 0} [Ax]_i / x_i$
with $\Gamma = \{x | x \geq 0, x \neq 0\}$
\end{itemize}
\end{thm}
